\section{Herstellung der ersten Platine}

\subsection{Entwurf einer Leiterplatte}
Zum Entwurf einer funktionsfähigen Platine sind ein Schaltplan, ein PCB-Layout und die passenden Bauteile mit zugehörigen Footprints erforderlich.
Für die erste Platine werden nur Testpunkte erstellt, weshalb die Erstellung eigener Bauteile und Footprints in sogenannten Libraries nicht im Fokus stand.
Der Schwerpunkt liegt vielmehr auf der Herstellung einer spezifischen Leiterbahn mit auf der Platine.\\
\\
Eine besondere Anforderung bestand darin, eine Leiterbahn mit einem definierten Widerstand von genau $200,\text{m}\Omega$ zu realisieren.
Um dies zu erreichen wird eine feste Breite für die Leiterbahn vorgegeben, anhand derer die erforderliche Länge der Bahn berechnet werden muss.

\paragraph{Berechnung:} 
\begin{align*}
\text{Gegeben:} & \quad b=0{,}275\,\text{mm}, \quad R=0{,}2\,\Omega, \quad \rho=0{,}01721\,\Omega\cdot\text{mm}^2/\text{m}, \quad h=0{,}035\,\text{mm} \\ 
\text{Gesucht:} & \quad l \\ R &= \frac{\rho \cdot l}{A} \quad \Rightarrow \quad l = \frac{R \cdot b \cdot h}{\rho} \\
l &= \frac{0{,}2\,\Omega \cdot 0{,}275\,\text{mm} \cdot 0{,}035\,\text{mm}}{0{,}01721\,\Omega\cdot\text{mm}^2/\text{m}} = 111{,}85\,\text{mm} 
\end{align*}
\\
Die berechnete Länge wird anschließend in das Platinenlayout erstellt.
Da die Gesamtlänge der Leiterbahn durch das Layout die berechneten Länge überschreiten könnte, wird die Leiterbahn in einem wellenartigen Muster umgesetzt, um die geforderte Gesamtlänge einzuhalten.

\begin{figure}[h]
\centering 
\includegraphics [width=\linewidth, height=6cm]{\figdir/PCB-Layout breite 0,275mm.png}
\caption{PCB Layout der Platine mit Leiterbahn = 0,275 mm}
\label{fig:Abbildung 1}
\end{figure}

\noindent
Außerdem wird der Name des Studierenden zusammen mit der Jahreszahl auch als Kupferbahn ausgeführt.
Bei der herkömmlichen Fertigung von Leiterplatten wird ein Bestückungsdruck für Bezeichnung der Komponenten und zusätzlichen Informationen zu den Bauteilen verwendet.
Da hier allerdings nur eine Kupferschicht vorgesehen ist, wird darauf verzichtet.
\\
Für die nachfolgende Messung bzw. Überprüfung unter dem Mikroskop werden vier Linien horizontal und vertikal in festem Abstand auf der Leiterplatte definiert.

\begin{figure}[h]
\centering 
\includegraphics [width=12cm ,height=6cm]{\figdir/Mikroskopie_paralleleLeiterbahn}
\caption{parallele Leiterbahnen unter Mikroskop}
\label{fig:Abbildung 2}
\end{figure}

\noindent
Die Ursachen für Fehler auf den geätzten Leiterbahnen sind vielschichtig.
So können beispielsweise Ungenauigkeiten beim Belichten oder Staub auf der Maske zu unpräzisen Strukturen führen.
Während des Entwicklungsprozesses können ungleichmäßige Ergebnisse durch unvollständige oder fehlerhafte Prozesse entstehen.
Im Ätzprozess selbst können Probleme wie unzureichende Ätzlösung, Lufteinschlüsse, Überätzung oder ungleichmäßiger Materialabtrag auftreten.
Um derartige Fehler zu vermeiden, bedarf es einer sauberen Arbeitsumgebung sowie optimierter Prozessparameter.

\subsection{Herstellung der Platine}
Für die Herstellung von Leiterplatten existieren diverse Verfahren.
An der Hochscule wird das Ätzverfahren eingesetzt wird.
Dieses besteht aus den drei Hauptschritten Belichten, Entwickeln und Ätzen.
Darüber hinaus sind mehrere Vor- und Nachbearbeitungsschritte erforderlich, um eine optimale Verarbeitung zu gewährleisten.\\
\\
Um das Material effizient zu nutzen und Produktionsabfälle zu minimieren, werden mehrere Leiterplatten zu einer Einheit zusammengefasst, die als Nutzen bezeichnet wird.
Ein solcher Nutzen entspricht der Größe einer Europlatine, was eine standardisierte und platzsparende Anordnung ermöglicht.\\
\\
Die Belichtungsmaske für den Nutzen wird durch Bedrucken einer transparenten Folie mit einem Laserdrucker erstellt.
Um eine gleichmäßige Lichtundurchlässigkeit sicherzustellen, werden zwei identische Folien exakt übereinander geklebt.
Dies gewährleistet eine hohe Präzision beim Belichten und damit eine gute Qualität der späteren Leiterplatten.\\
\\
Nach dem Ätzen sind die Platinen noch als Teil des Europlatinen-Nutzens verbunden und müssen im Anschluss voneinander getrennt werden.
Der Trennvorgang erfolgt mithilfe einer Hebelschere.
Die Verwendung dieses Werkzeugs ermöglicht ein sauberes und präzises Zuschneiden der einzelnen Platinen.\\

\subsubsection{Belichten}
Beim Ätzverfahren befindet sich auf der Platine ein Ätzresistlack. Dieser Lack muss mit Hilfe von Belichten an bestimmten Stellen seine Wirkung verlieren. \ldots

\subsubsection{Entwickeln}
Für die Entwicklung der Platine wird diese in einen Rahmen eingespannt und anschließend in eine Kammer gehängt. In dieser Kammer wird der Rohling mit Natronlauge besprüht. \ldots

\subsubsection{Ätzen}
Zum Ätzen der Platine wird diese wieder in einen Rahmen gespannt und anschließend in einer Kammer mit Eisen-III-Chlorid besprüht. \ldots
