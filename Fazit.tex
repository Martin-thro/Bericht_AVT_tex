
\section{Fazit}
Das Praktikum in der Aufbau- und Verbindungstechnik ist eine spannende und lehrreiche Erfahrung, die mir einen tiefen Einblick in die Herstellung und das Design von Leiterplatten gegeben hat.
\\
\\
Von den ersten Schritten des Entwurfs einer einfachen Leiterplatte bis hin zur komplexen Schaltplan- und Layouterstellung in KiCad werden alle wesentlichen Aspekte detailliert vermittelt.
Auch wenn der Prozess der Leiterplattenherstellung an der Hochschule nicht immer reibungslos verläuft, haben mir gerade diese Herausforderungen gezeigt, wie viele Faktoren in der Aufbau- und Verbindungstechnik zu berücksichtigen sind.
\\
\\
Die Bestellung von Leiterplatten bei einem externen Hersteller ist von besonderem Interesse.
Dadurch ist es möglich, wertvolle Einblicke in die Materialbestellung und -auswahl zu erlangen.
\\
\\
Der KiCad-Workshop ist ein zentraler Bestandteil des Praktikums und gibt Einblicke in die Bedeutung und den Umgang mit Bibliotheken, Bauteilen und Footprints.
Präzises Arbeiten beim Layout und die sorgfältige Platzierung der Bauteile sind entscheidend für die Anforderungen an eine funktionsfähige Leiterplatte. 
\\
\\
Auch die praktischen Übungen beim Bestücken und Löten verdeutlichten die Komplexität und die Notwendigkeit präzisen Arbeitens in der Fertigung.
Insbesondere die Herausforderungen im Umgang mit der Lötpaste und Bestückungsautomaten sind sehr lehrreich.
\\
\\
Insgesamt vermittelte das Praktikum einen umfassenden Einblick in den Entwurf und die Fertigung einer Leiterplatte. Die gewonnenen Erkenntnisse und Erfahrungen bilden eine solide Grundlage für zukünftige Projekte und Arbeiten im Bereich der Leiterplattentechnik.
