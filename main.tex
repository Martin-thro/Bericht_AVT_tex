
\documentclass[a4paper,12pt]{article}
\usepackage[utf8]{inputenc}
\usepackage[ngerman]{babel}
\usepackage{graphicx}
\usepackage{hyperref}
\usepackage{amsmath}
\usepackage{amssymb}
\usepackage{geometry}
\geometry{a4paper, margin=2.5cm}


% Erstellen der directorys für Bilder und Tabellen
\def\figdir{figures}
\def\tabledir{tables}


\begin{document}

\newpage
\include{title}

\newpage
\tableofcontents		% Inhaltsverzeichnis
\newpage

% Include sections

\section*{Einleitung}
Die Entscheidung für das FWPM Aufbau- und Verbindungstechnik resultierte aus dem Wunsch, mein technisches Verständnis im Bereich der Elektronik und deren Herstellung zu vertiefen.
\\
\\
Das Praktikum war eine interessante Möglichkeit, die Entwicklung einer Leiterplatte von der ursprünglichen Idee bis zur endgültigen Realisierung zu verfolgen.
Insbesondere die Inhalte aus den Präsentationen konnte im Praktikum gut ergänzt werden.
\\
\\
Durch diese Erkenntnis konnte ich meine Kompetenzen im Bereich der Leiterplattentechnik und im Umgang mit den Maschinen bei der Herstellung dieser deutlich ausbauen.

\newpage
\section{Herstellung der ersten Platine}

\subsection{Entwurf einer Leiterplatte}
Zum Entwurf einer funktionsfähigen Platine sind ein Schaltplan, ein PCB-Layout und die passenden Bauteile mit zugehörigen Footprints erforderlich.
Für die erste Platine werden nur Testpunkte erstekkt, weshalb die Erstellung eigener Bauteile und Footprints in sogenannten Libraries nicht im Fokus stand.
Der Schwerpunkt liegt vielmehr auf der Herstellung einer spezifischen Leiterbahn mit auf der Platine.\\
\\
Eine besondere Anforderung bestand darin, eine Leiterbahn mit einem definierten Widerstand von genau $200,\text{m}\Omega$ zu realisieren.
Um dies zu erreichen wird eine feste Breite für die Leiterbahn vorgegeben, anhand derer die erforderliche Länge der Bahn berechnet werden muss.

\paragraph{Berechnung:} 
\begin{align*}
\text{Gegeben:} & \quad b=0{,}275\,\text{mm}, \quad R=0{,}2\,\Omega, \quad \rho=0{,}01721\,\Omega\cdot\text{mm}^2/\text{m}, \quad h=0{,}035\,\text{mm} \\ 
\text{Gesucht:} & \quad l \\ R &= \frac{\rho \cdot l}{A} \quad \Rightarrow \quad l = \frac{R \cdot b \cdot h}{\rho} \\
l &= \frac{0{,}2\,\Omega \cdot 0{,}275\,\text{mm} \cdot 0{,}035\,\text{mm}}{0{,}01721\,\Omega\cdot\text{mm}^2/\text{m}} = 111{,}85\,\text{mm} 
\end{align*}
\\
Die berechnete Länge wird anschließend in das Platinenlayout erstellt.
Da die Gesamtlänge der Leiterbahn durch das Layout die berechneten Länge überschreiten konnte, wird die Leiterbahn in einem wellenartigen Muster umgesetzt, um die geforderte Gesamtlänge einzuhalten.\\

\begin{figure}[h]
\centering 
\includegraphics [width=\linewidth, height=6cm]{\figdir/PCB-Layout breite 0,275mm.png}
\caption{PCB Layout der Platine mit Leiterbahn = 0,275 mm}
\label{fig:Abbildung 1}
\end{figure}




\subsection{Herstellung der Platine}
Zur Herstellung von Platinen gibt es verschiedenste Verfahren. An der Hochschule werden die Platinen mittels eines Ätzverfahrens hergestellt. Dieses Verfahren besteht im Wesentlichen aus drei Schritten: dem Belichten, Entwickeln und Ätzen.

\subsubsection{Belichten}
Beim Ätzverfahren befindet sich auf der Platine ein Ätzresistlack. Dieser Lack muss mit Hilfe von Belichten an bestimmten Stellen seine Wirkung verlieren. \ldots

\subsubsection{Entwickeln}
Für die Entwicklung der Platine wird diese in einen Rahmen eingespannt und anschließend in eine Kammer gehängt. In dieser Kammer wird der Rohling mit Natronlauge besprüht. \ldots

\subsubsection{Ätzen}
Zum Ätzen der Platine wird diese wieder in einen Rahmen gespannt und anschließend in einer Kammer mit Eisen-III-Chlorid besprüht. \ldots

\newpage
\section{KiCad Workshop}

\subsection{Libraries}
Um Leiterplatten zu entwerfen, sind Bauteile und zugehörige Footprints unverzichtbar.
Bauteile und Footprints werden in Libraries oder auch Bibliotheken angelegt und eingepflegt.
Man kann zu jedem Projekt eine eigene Library anlegen oder eine allgemeine Library verwenden, dies bleibt dem Entwickler selbst überlassen. 

In der freien Wirtschaft jedoch wird von Firmen häufig vorgeschrieben, wie und unter welchen Bedingungen Bauteile in das System einzupflegen sind.
Das Anlegen einer eigenen Bauteilbibliothek hat den Vorteil, dass sich der Anwender schneller zurechtfindet.
Oft wird auch nur ein gewisses Kontingent an Bauteilen verwendet, da der jeweilige Hersteller der Leiterplatte nur ein definiertes Sortiment nutzt. 

\subsection{Bauteile}
Um eine Leiterplatte fertigen zu können, braucht man einen Schaltplan, nach welchem man vorgeht.
Ohne Schaltung braucht man auch keine Leiterplatte.
Um wiederum einen Schaltplan erstellen zu können, benötigt man Bauteile.
Diese Bauteile sind zum Teil in KiCad-Standardbibliotheken bereits vorhanden.

Im Falle des Taschenlampenprojekts wurden jedoch alle Bauteile extra in einer eigenen Library angelegt.
Die Bauteile erfüllen primär die Rolle, den Schaltplan richtig zu zeichnen und auf Funktion zu prüfen.
Bauteile werden im Bauteileditor gezeichnet und dann in einer Library gespeichert.
In den Symboleigenschaften können zusätzlich Artikelnummern, Datenblätter und 3D-Modelle hinterlegt werden.

Bauteile können im Bauteileditor von Grund auf neu gezeichnet werden, oder man kopiert bereits vorhandene Bauteile und ändert diese entsprechend den Anforderungen ab.

\subsection{Footprints}
Um von den Bauteilen eine Verbindung zur Leiterplatte herstellen zu können, müssen sogenannte Footprints erstellt werden.
Der Footprint oder auch Fußabdruck der Leiterplatte wird je nach Bauteilstruktur so aufgebaut, dass alle Kontaktierungen auf die Leiterplatte passen. 

Die Kontaktherstellung zur Leiterplatte erfolgt über sogenannte Lötpads.
Diese sind kleine Kupferfelder auf der Leiterplatte, auf welche die Kontakte der Bauteile in der Bestückung aufgelötet werden. Beim Zeichnen der Lötpads sind einige Aspekte zu beachten, wie etwa, ob das Bauteil per Hand oder in einem automatischen Prozess gelötet wird. 

Beim Löten per Hand sollten die Pads etwas größer ausgeführt werden, da man per Hand nicht so genau löten kann.
Bei automatischen Prozessen sollten die Pads kleiner gehalten werden, um Platz zu sparen und Fehler wie den sogenannten Gravestone-Effekt zu vermeiden, bei dem sich Bauteile aufstellen wie Grabsteine. 

\newpage

\section{Bestücken der Platinen}

\subsection{Bestücken und Löten der Taschenlampe}
Die Taschenlampe wurde extra so entworfen, dass auch die SMD-Bauteile per Hand aufgelötet werden können. \ldots

\subsection{Bestücken und Löten des Radios}
Die Radioplatine wurde mit verschiedenen Verfahren bestückt. \ldots

\newpage

\section{Fazit}
Das Praktikum in der Aufbau- und Verbindungstechnik bot eine umfassende Einführung in die Herstellung und das Design von Leiterplatten. \ldots


\newpage
\section*{Abbildungsverzeichnis}
\begin{itemize}
    \item Abbildung 1: Layout der Platine Leiterbahnbreite 0,3mm
    \item Abbildung 2: Schaltplan Messpunkte beschriftet und Widerstandswert angegeben.
    
\end{itemize}

\section*{Quellenverzeichnis}
\begin{itemize}
    \item Quelle 1: \url{https://www.pcbway.com/blog/
    PCB_Manufacturing_Information/What_is_PCB_Tombstone_.html}
    \item Quelle 2: \url{https://learning-campus.th-rosenheim.de/
    course/view.php?id=5012}
\end{itemize}

\end{document}
